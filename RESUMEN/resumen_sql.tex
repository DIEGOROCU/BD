\documentclass[10pt,a4paper,twocolumn]{article}
\usepackage[utf8]{inputenc}
\usepackage[spanish]{babel}
\usepackage[margin=1.5cm]{geometry}
\usepackage{listings}
\usepackage{xcolor}
\usepackage{enumitem}

% Configuración de colores para código
\definecolor{codegreen}{rgb}{0,0.6,0}
\definecolor{codegray}{rgb}{0.5,0.5,0.5}
\definecolor{codepurple}{rgb}{0.58,0,0.82}
\definecolor{backcolour}{rgb}{0.95,0.95,0.92}

\lstdefinestyle{sqlstyle}{
    backgroundcolor=\color{backcolour},
    commentstyle=\color{codegreen},
    keywordstyle=\color{blue},
    numberstyle=\tiny\color{codegray},
    stringstyle=\color{codepurple},
    basicstyle=\ttfamily\scriptsize,
    breakatwhitespace=false,
    breaklines=true,
    captionpos=b,
    keepspaces=true,
    showspaces=false,
    showstringspaces=false,
    showtabs=false,
    tabsize=2
}

\lstset{style=sqlstyle}

\setlist[itemize]{noitemsep, topsep=0pt, leftmargin=*}
\setlength{\columnsep}{1cm}
\setlength{\parindent}{0pt}
\setlength{\parskip}{2pt}

\title{\textbf{Resumen SQL/DES}}
\author{}
\date{}

\begin{document}

\maketitle

\section*{1. Comandos de Sistema DES}

\textbf{/abolish}: Elimina todas las tablas y vistas.\\
\textbf{/multiline on}: Permite consultas multilínea.\\
\textbf{/duplicates off}: Elimina duplicados automáticamente.

\section*{2. DDL - Definición de Datos}

\textbf{CREATE TABLE}: Crea una nueva tabla.
\begin{lstlisting}
create table tabla(
  col1 tipo primary key,
  col2 tipo,
  col3 tipo
);
\end{lstlisting}

\textbf{Tipos}: \texttt{string}, \texttt{int}, \texttt{float}, etc.\\
\textbf{PRIMARY KEY}: Define clave primaria (simple o compuesta).

\section*{3. DML - Manipulación de Datos}

\textbf{INSERT}: Inserta tuplas en una tabla.
\begin{lstlisting}
insert into tabla 
  values('val1','val2',123);
insert into tabla(col1,col2) 
  values('val1','val2');
\end{lstlisting}

\textbf{NULL}: Representa valores desconocidos/no aplicables.

\section*{4. Consultas SELECT}

\textbf{SELECT básico}:
\begin{lstlisting}
select columnas from tabla
  where condicion;
\end{lstlisting}

\textbf{SELECT *}: Selecciona todas las columnas.\\
\textbf{DISTINCT}: Elimina duplicados en resultados.

\section*{5. Operadores Lógicos}

\textbf{AND, OR, NOT}: Operadores booleanos.\\
\textbf{Comparación}: =, <>, <, >, <=, >=\\
\textbf{IS NULL / IS NOT NULL}: Comprueba valores nulos.\\
\textbf{IN / NOT IN}: Pertenencia a conjunto.

\section*{6. VISTAS (CREATE VIEW)}

Las vistas son consultas almacenadas que pueden referenciarse como tablas.

\begin{lstlisting}
create view nombre_vista as
  select ... from ... where ...;
\end{lstlisting}

\textbf{Ventajas}: Reutilización, abstracción, simplicidad.

\section*{7. Operaciones de Conjuntos}

\textbf{UNION}: Unión de resultados (elimina duplicados).
\begin{lstlisting}
select dni from programadores
union
select dni from analistas;
\end{lstlisting}

\textbf{UNION DISTINCT}: Unión eliminando duplicados explícitamente.

\textbf{INTERSECT}: Intersección de resultados.
\begin{lstlisting}
select dni from programadores
intersect
select dni from analistas;
\end{lstlisting}

\textbf{EXCEPT}: Diferencia de conjuntos (A - B).
\begin{lstlisting}
select dni from programadores
except
select dni from analistas;
\end{lstlisting}

\textbf{EXCEPT ALL}: Diferencia manteniendo duplicados.

\section*{8. JOIN - Combinación de Tablas}

\textbf{JOIN implícito} (producto cartesiano con WHERE):
\begin{lstlisting}
select * from tabla1, tabla2
  where tabla1.id = tabla2.id;
\end{lstlisting}

\textbf{NATURAL JOIN}: Join por columnas con mismo nombre.
\begin{lstlisting}
select * from tabla1 
  natural join tabla2;
\end{lstlisting}

\textbf{JOIN explícito}:
\begin{lstlisting}
select * from tabla1 
  join tabla2 on cond;
\end{lstlisting}

\section*{9. Funciones de Agregación}

\textbf{SUM(col)}: Suma de valores.\\
\textbf{COUNT(*)}: Cuenta tuplas.\\
\textbf{AVG(col)}: Media aritmética.\\
\textbf{MAX(col)}: Valor máximo.\\
\textbf{MIN(col)}: Valor mínimo.

\begin{lstlisting}
select dni, sum(horas) as total
from distribucion
group by dni;
\end{lstlisting}

\section*{10. GROUP BY y HAVING}

\textbf{GROUP BY}: Agrupa filas por columna(s).
\begin{lstlisting}
select col, count(*) from tabla
  group by col;
\end{lstlisting}

\textbf{HAVING}: Filtra grupos (después de GROUP BY).
\begin{lstlisting}
select col, count(*) from tabla
  group by col
  having count(*) > 5;
\end{lstlisting}

\textit{Diferencia}: WHERE filtra antes de agrupar, HAVING después.

\section*{11. Subconsultas}

\textbf{Subconsulta escalar}: Devuelve un único valor.
\begin{lstlisting}
select * from tabla where col =
  (select max(col) from tabla);
\end{lstlisting}

\textbf{Subconsulta en IN}:
\begin{lstlisting}
where dni in 
  (select dni from analistas);
\end{lstlisting}

\textbf{Subconsulta en FROM}: Tabla derivada.
\begin{lstlisting}
select * from 
  (select dni from prog 
   union select dni from anal);
\end{lstlisting}

\section*{12. DIVISION}

Operador específico de DES para encontrar elementos que se relacionan con todos los de otro conjunto.

\begin{lstlisting}
select dniEmp from
(select codigoPr,dniEmp 
 from distribucion)
division
(select codigoPr from dist
 where dniEmp='5');
\end{lstlisting}

\textbf{Interpretación}: Empleados asignados a todos los proyectos del empleado '5'.

\section*{13. Operaciones Aritméticas}

Se pueden realizar operaciones en SELECT:
\begin{lstlisting}
select codigo, horas*1.2 
  from distribucion;
\end{lstlisting}

\textbf{Operadores}: +, -, *, /

\section*{14. Alias y RENAME}

\textbf{AS}: Define alias para columnas/tablas.
\begin{lstlisting}
select dni as identificador,
  sum(horas) as total
from distribucion;
\end{lstlisting}

\textbf{RENAME} (sintaxis DES):
\begin{lstlisting}
rename (dniEmp as dni)
  (distribucion);
\end{lstlisting}

\section*{15. Consultas Complejas}

\textbf{Combinando múltiples operaciones}:
\begin{lstlisting}
create view vista_compleja as
(select dni,nombre,codigoPr 
 from (select * from prog 
       union select * from anal)
 join dist on dni = dniEmp)
union
(select dni,nombre,codigo 
 from (select * from prog 
       union select * from anal)
 join proy on dni = dniDir);
\end{lstlisting}

\section*{16. Buenas Prácticas}

\begin{itemize}
\item Usar vistas para consultas complejas reutilizables
\item Aplicar filtros WHERE antes de JOIN cuando sea posible
\item Usar DISTINCT solo cuando sea necesario (impacto en rendimiento)
\item Nombrar columnas con alias descriptivos
\item Dividir consultas complejas en vistas intermedias
\item Considerar NULL en comparaciones (NULL != NULL)
\end{itemize}

\section*{17. Orden de Ejecución SQL}

Orden lógico de evaluación:
\begin{enumerate}[noitemsep]
\item FROM (tablas/joins)
\item WHERE (filtrado de filas)
\item GROUP BY (agrupación)
\item HAVING (filtrado de grupos)
\item SELECT (proyección)
\item UNION/INTERSECT/EXCEPT
\item ORDER BY (no usado en ejemplos)
\end{enumerate}

\section*{18. Ejemplos Prácticos}

\textbf{Empleados sin teléfono}:
\begin{lstlisting}
select dni,nombre from prog
where telefono is null
union
select dni,nombre from anal
where telefono is null;
\end{lstlisting}

\textbf{Suma de horas por empleado}:
\begin{lstlisting}
select dni, sum(horas) as total
from (select dni from prog 
      union select dni from anal)
join dist on dni = dniEmp
group by dni;
\end{lstlisting}

\textbf{Proyectos sin analistas}:
\begin{lstlisting}
select codigo from proyectos
except
select distinct codigo from
(select codigoPr as codigo,
        dniEmp from dist)
join analistas on dniEmp = dni;
\end{lstlisting}

\end{document}
